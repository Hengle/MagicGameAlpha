\documentclass[12pt,openright]{book}
%%%%%%%%%%%%%%%%%%%%%%%%%%%%%%%%%%%%%%%%%%%%%%%%%%%%%%%%%%%%%%%%%%
% 引用参考文献为上标表示使用命令'\citeu'  by Liang Jin
\newcommand{\citeu}[1]{$^{\mbox{\protect \scriptsize \cite{#1}}}$}
%%%%%%%%%%%%%%%%%%%%%%%%%%%%%%%%%%%%%%%%%%%%%%%%%%%%%%%%%%%%%%%%%%
\usepackage{amssymb}
\usepackage{amsmath}
\usepackage{graphicx}
\usepackage{bm}
\usepackage{epsf}
%\usepackage{eucal}
\usepackage{ifxetex}
\usepackage[numbers,sort&compress]{natbib}% 此命令要放在设置hyperref的命令之前

\usepackage{graphicx}  % needed for figures
\usepackage{float}
\usepackage{dcolumn}   % needed for some tables
\usepackage{bm}        % for math
\usepackage{amssymb}   % for math
\usepackage{slashed}   % for Dirac Slash
\usepackage{amsmath}   % for mutiline eqn
\usepackage{simplewick} % for contraction
\usepackage{verbatim}  % for multi-line comment
\usepackage{color} % for textcolor
\usepackage{appendix} % for appendix

%\RequirePackage[dvipdfm]{hyperref}
%   根据需要选择 hyperref 宏包调用方式.
\ifxetex
  \usepackage[bookmarksnumbered]{hyperref}
\else
  \usepackage[unicode,bookmarksnumbered]{hyperref}
\fi
%%%%%%%%%%%%%%%%%%%%%%%%%%%%%%%%%%%%%%%%%%%%%%%%%%%%%%
%更改超链接的颜色 by Liang Jin
\hypersetup{colorlinks,citecolor=blue,linkcolor=blue,urlcolor=blue,CJKbookmarks=blue}
%\hypersetup{colorlinks,citecolor=red,linkcolor=red,urlcolor=red,CJKbookmarks=red}
%\hypersetup{colorlinks,citecolor=blue,linkcolor=magenta,urlcolor=cyan}
%%%%%%%%%%%%%%%%%%%%%%%%%%%%%%%%%%%%%%%%%%%%%%%%%%%%%%
\usepackage[emptydoublepage]{gameThesis}
%\usepackage{biborder}% biborder 宏包使得参考文献按照出现顺序排序. 如果您使用 bib文件, %不需要此宏包.
\newtheorem{Theorem}{\hskip 2em 定理}[chapter]
\newtheorem{Lemma}[Theorem]{\hskip 2em 引理}
\newtheorem{Corollary}[Theorem]{\hskip 2em 推论}
\newtheorem{Proposition}[Theorem]{\hskip 2em 命题}
\newtheorem{Definition}[Theorem]{\hskip 2em 定义}
\newtheorem{Example}[Theorem]{\hskip 2em 例}
\renewcommand\appendixname{附录}
%\renewcommand{\proofname}{\bf 证明}
\graphicspath{{figures/}}
%%%%%%%%%%%%%%%%%%%%%%%%%%%%%%%%%%%%%%%%%%%%%%%
%%%%%%%%%%%%%%%%%%%%%%%%%%%%%%%%%%%%%%%%%%%%%%%
%加入对图表的旋转操作 by Liang Jin
\usepackage[figuresright]{rotating}
%%%%%%%%%%%%%%%%%%%%%%%%%%%%%%%%%%%%%%%%%%%%%%%
%避免修改mathcal字体 by Liang Jin
%\usepackage[utopia]{mathdesign}
\DeclareSymbolFont{symbols}{OMS}{cmsy}{m}{n}
\DeclareSymbolFontAlphabet{\mathcal}{symbols}
%%%%%%%%%%%%%%%%%%%%%%%%%%%%%%%%%%%%%%%%%%%%%%%
\begin{document}

%  设置基本信息
%  注意:  逗号`,'是项目分隔符. 如果某一项的值出现逗号, 应放在花括号内, 如 {,}
%
\NKTsetup{}
%%%%%%%%%%%%%%%%%%%%%%%%%%%%%%
%% 前言部分
%%%%%%%%%%%%%%%%%%%%%%%%%%%%%%
%更改正文前页码缺省设置:小写罗马->大写罗马 by Liang Jin
\makeatletter
\renewcommand\frontmatter[1]{%
   \cleardoublepage
 \@mainmatterfalse
 \pagenumbering{#1}}
\makeatother
\frontmatter{Roman}
%%%%%%%%%%%%%%%%%%%%%%%%%%%%%%
%%%%%%%%%%%%%%%%%%%%%%%%%%%%%%%%%%%%%%%%%%%%%%%%%%%%%%%%%%%%
%%下面代码消除索引图表中章节间的空行 by Liang Jin
\makeatletter
\def\@chapter[#1]#2{\ifnum \c@secnumdepth >\m@ne
                       \if@mainmatter
                         \refstepcounter{chapter}%
                         \typeout{\@chapapp\space\thechapter}%
                         \addcontentsline{toc}{chapter}%
                                   {\protect\numberline{\chaptername}\hspace{0em}#1}%
                       \else
                         \addcontentsline{toc}{chapter}{#1}%
                       \fi
                    \else
                      \addcontentsline{toc}{chapter}{#1}%
                    \fi
                    \chaptermark{#1}%
                    \if@twocolumn
                      \@topnewpage[\@makechapterhead{#2}]%
                    \else
                      \@makechapterhead{#2}%
                      \@afterheading
                    \fi}
\makeatother
%%%%%%%%%%%%%%%%%%%%%%%%%%%%%%%%%%%%%%%%%%%%%%%%%%%%%%%%%%%%
%%%%%%%%%%%%%%%%%%%%%%%%%%%%%%%%%%%%%%%%%%%%%%%%%%%%%%%%%%%%
% 在目录中加入目录的链接 by Liang Jin
%  \addcontentsline{toc}{chapter}\contentsname
%  \tableofcontents
  % 摘要
\phantomsection
%\addcontentsline{toc}{chapter}{目~~录}
\tableofcontents
  % 插图目录
\cleardoublepage
\phantomsection
%\addcontentsline{toc}{section}{表格目录}
%\listoftables %显示所有表格目录
%%%%%%%%%%%%%%%%%%%%%%%%%%%%%%
%% 正文部分
%%%%%%%%%%%%%%%%%%%%%%%%%%%%%%
\mainmatter
    \chapter{美术资源规范}

bla, bla....

所以说,还是latex好!

\section{角色系统}

\subsection{地面人形角色}

\subsection{地面四脚角色}

\subsection{空中人形角色}

\subsection{空中怪物角色}

\subsection{塔状建筑角色}

\section{场景系统}

阿拉拉拉!



    \chapter{角色定制系统}

bla, bla....

所以说,还是latex好!


%%%%%%%%%%%%%%%%%%%%%%%%%%%%%%
\backmatter
% 附录
\end{document}
